\documentclass{fkssolpub}

\usepackage[czech]{babel}
\usepackage{fontspec}

\author{Ondřej Sedláček}
\school{Gymnázium Oty Pavla} 
\series{3}
\problem{3} 

\begin{document}

\section{Pro souvislou řadu šelem}

Nejdříve vyřešíme případ, kdy stojící šelmy tvoří
souvislou řadu. Na to můžeme využít Goldbachovi
hypotézi, která říká, že jakékoli sudé číslo větší než
dva lze vyjádřit jako součet dvou prvočísel. Přestože
na ní neexistuje formální důkaz, s největší 
pravděpodobností bude do deseti milionů platit.

Musíme však ukázat, že bude platit, přestože v seznamu
prvočísel nemůžeme mít číslo dva. Na důkaz stačí jen
ukázat, že jediné číslo, které v součtu bude mít dvojku,
je číslo 4, protože jinak by nemohl být součet sudý.

Této hypotézi můžeme využít i u lichých čísel. Z ní
pak vyplývá, že všechna lichá čísla, které nejsou
prvočísly, lze vyjádřit jako součet sudého čísla a
trojky. Proto liché složené číslo lze být vyjádřeno
jako součet tří čísel. Taky u nich nejde mít počet
sčítanců menší, protože součet dvou lichých čísel je
sudé číslo.

Tím máme pokryté skoro všechna čísla až na 1, 2, 4.
Tyhle čísla lze vyjádřit jako rozdíl.

Číslo 2 máme už vyřešené ze zadání. Lze vyjádřit jako
$2 = 5 - 3$, což musí být optimální řešení, jelikož
jinak by 2 musela být povolená hlasitost, což není.

Číslo 1 vyjádříme jako $1 = 7 - 3 - 3$. Důvod optimálnosti
je stejný jako pro liché složené čísla a zároveň 1 není
prvočíslo.

Spolu s číslem 4 vyřeším první vstup. Pro něj bude
výstup \verb|2|, protože 4 lze vyjádřit jako
$4 = 7 - 3$.

Postup je následující: nejdříve se přečte vstup a 
spočítá se velikost souvislé řady, následně vrátíme
\verb|2|, kdyby se jednalo o sudé číslo, jinak
zkontrolujeme, jestli se jedná o prvočíslo. Pak
vrátíme \verb|1|, pokud by šlo o prvočíslo, jinak
vrátíme \verb|3|. Čtení vstupu a inicializace má
složitost $\mathcal{O}(N)$ a kontrola, jestli se
jedná o prvočíslo, $\mathcal{O}(\sqrt{N})$, proto
celková časová složitost s tímto postupem je
$\mathcal{O}(N)$.

\section{Pro nesouvislou řadu šelem}

Pro nesouvislou řadu stojících šelem postup výše
nemůžeme použít. Můžeme však použít při řešení
následující algoritmus. Souvislou řadou budu mínit
šelmy, které jsou ve stejném stavu a zároveň jsou
pospolu.

Nejdříve přečteme vstup a z něj vytvoříme pole, které
bude ukládat délku souvislých řad 
(pro vstup \verb|25 26 28| to je \verb|[2, 1, 1]|).
Při tom zjistíme vzdálenost první stojící šelmy po poslední
stojící šelmu a po toto číslo najdeme všechna prvočísla
pomocí Erathosthenově sítu, což bude trvat maximálně 
$\mathcal{O}((x_N - x_1) \cdot \log (\log (x_N - x_1)))$. 
Samozřejmě pak odstraníme dvojku ze seznamu prvočísel. 

Pak pro každý prvek v seznamu ukládající délky souvislých
řad zjistíme stejným postupem
jako první části, kolik zařvání potřebujeme, aby
šelmy v této řadě se postavili či sedli a aby ostatní
šelmy zůstali ve stejném stavu. Pak během toho, co řada
stojících šelem není souvislá, zkoušíme pro každou 
souvislou řadu, jakým směrem a jakou hlasitostí je
nejlepší zařvat (pro směr doprava voláme od první šelmy
v souvislé řadě, pro směr doleva od poslední), vybereme
ten, který nejvíce zminimalizuje počet zařvání, upravíme
seznam a opakujeme. Tento postup má maximální složitost
$\mathcal{O}(N^2 \cdot \pi(x_N - x_1) \log \pi(x_N - x_1))$,
kde $\pi(n)$ je hustota prvočísel do čísla $n$.

Jakmile už máme souvislou řadu, vyhodnotíme v čase
$\mathcal{O}(\log \pi(x_N - x_1))$, kolik zařvání
potřebujeme na posazení těchto šelem, a vrátíme
celkový výsledek. Pokud použijeme substituci
$a = x_N - x_1$, celkový čas je $\mathcal{O}(
N^2 \cdot \pi(a) \log \pi(a) + a \log \log a
)$.

Tímto způsobem můžeme vyřešit druhou úlohu. Ze vstupu
vytvoříme seznam \verb|[1, 5, 3, 6, 1]| a upravujeme
takto:

\begin{verbatim}
[1, 5, 3, 6, 1] // od poslední šelmy doprava s hlasitostí 7
  ->
[1, 5, 9] // postavíme ty uprostřed
  ->
[15] // a máme souvislou posloupnost
\end{verbatim}

Protože 15 je liché složené číslo, výsledek druhé úlohy
je \verb|5|.

\end{document}
