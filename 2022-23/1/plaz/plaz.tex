\documentclass{fkssolpub}

\usepackage[czech]{babel}
\usepackage{fontspec}

\usepackage{fkssugar}
 
\author{Ondřej Sedláček}
\school{Gymnázium Oty Pavla} 
\series{1}
\problem{1} 

\begin{document} 

\section{Bez prkna}

Pro nalezení cesty bez použití prkna stačí implementovat BFS pro graf
definovaný tabulkou. 

Protože v tomto algoritmu ukládáme informace o pláži do dvourozměrného
pole o velikosti $n \times n$, inicializace má časovou složitost $\mathcal{O}(n^2)$.

U BFS je časová asymptotická složitost $\mathcal{O}(V + E)$, tudíž
pro určení složitosti potřebujeme zjistit počet vrcholů a hran. 
Největší počet vrcholů může být $n^2$, protože tolik je počet buněk v
tabulce o rozměrech $n \times n$. Dále počet hran může být nejvíc $2n(n - 1)$,
protože $2(n - 1)$ je počet hranic oddělující jednotlivé sloupce a řádky
a $n$ je počet hran na určité hranici sloupců či řádků. Po dosazení nám
vyjde následující:

\[
  \mathcal{O}(V + E) = \mathcal{O}(n^2 + 2n(n - 1)) = \mathcal{O}(3n^2 - 2n)
   = \mathcal{O}(n^2)
\]

Časová asymptotická složitost algoritmu samotného je proto kvadratická.

Neboť v tomto algoritmu ukládám počet navštívených buněk a informace o 
pláži v dvourozměrném poli, musí být celková prostorová asymptotická 
složitost $\mathcal{O}(n^2)$.

\section{S prknem}

To, co by nás mohlo napadnout jako první, je to, že použijeme předchozí
algoritmus s tím rozdílem, že ve frontě se budou ukládat taky 
informace o možnosti použití prkna a při možnosti použití prkna 
vstopíme na políčko s pískem. Problém je však v tom, že může dojít k 
situaci, kdy se vlnou bez prkna uzavře vrchol a zablokuje tím průchod 
vlnám s prknem, které by mohli jít kratší cestou. Proto musíme 
jednotlivé navštívené vrcholy ukládat \textit{do samostatných polí} podle 
toho, jakým typem vlny byla navštívena.

Z toho můžeme rovnou zjistit, že BFS může proběhnout maximálně dvakrát a 
navíc budeme ukládat jen další pole navštívených vrcholů. Proto jak 
časová, tak i prostorová asymptotická složitost bude stejná, protože 
se složitosti tohoto a předchozího algoritmu liší jen o multiplikační 
konstantu, která se zanedbává.

\end{document}
