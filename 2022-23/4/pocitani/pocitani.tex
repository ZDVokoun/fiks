\documentclass{fkssolpub}

\usepackage[czech]{babel}
\usepackage{fontspec}
\usepackage{listings}

\author{Ondřej Sedláček}
\school{Gymnázium Oty Pavla} 
\series{4}
\problem{2} 

\lstset{
  breaklines=true,
  basicstyle=\ttfamily
}

\begin{document}

Veškeré symboly označující funkce jsem přeoznačil písmenami od a po
n podle toho, které z nich se v nápovědě vyskytli dříve.

\section{První část}

Už podle názvu a prvních rovností v první části tohoto úkolu
lze usoudit, že znaky budou s největší pravděpodobností
číslice. Zároveň nám na mysl může přijít domněnka, že tato čísla budou
seřazena od nejmenšího po největší. Jak se dále dozvíme, nejlépe
odpovídají číslice od 0 do 6.

Když se podíváme na rovnosti pod číslicemi, všimneme si, že by
mohli značit funkčních hodnot nějakých funkcí. První funkce by
mohla být $a(x) = x$, protože vůbec nemění hodnotu daných čísel.
U druhý pokud je první číslice tvaru X nula, pak se v této funkci
musí příčítat konstanta, proto řešením se zdá $b(x) = x + 1$. Musíme však
vyřešit poslední rovnost, protože tam se ze šestky stane nula.
Z toho vyplývá, že výsledná funkce bude $b(x) = (x + 1) \, \% \, 7$.
Podobně pak zjistíme, že $c(x) = 2x \, \% \, 7$ a 
$d(x, y) = (y + 1) \, \% \, 7$.
Díky těmto informacím můžeme vyřešit první nedořešenou sekci:

\begin{verbatim}
  a(a(b(4))) = 5 
  c(0) = 0 
  c(b(5)) = 5 
  d(a(b(b(c(a(6))))),b(6)) = 1 
\end{verbatim}

\section{Velká instance}

Tady je velmi užitečné si uvědomit několik vlatností těch funkcí,
které se ve velké arkádě nachází. Nejprve musí platit, že koeficienty
těch funkcí musí být nezáporná celá čísla, protože kdyby nebyla, pak by pro
nějakou hodnotu muselo nutně vyjít takové číslo, které bychom
nebyli jejich zápisem schopni vyjádřit. Dále se znalostí modulární
aritmetiky můžeme zjistit, že nám stačí hledat jen koeficienty menší
než 7.

Díky těmto vlastnostem si můžeme dovolit hledat koeficienty funkcí
hrubou silou. Postupoval jsem následujícím způsobem: nejprve jsem
nahradil veškeré vyřešené funkce jejich funkčními hodnotami pro daný
vstup a následně jsem substituoval z rovnic navzájem. Pak jsem si
vybral tu funkci, pro kterou jsem díky substitucím v daném kole
získal nejvíce funkčních hodnot, našel jsem její koeficienty a 
pro každé nalezené $n$-tice koeficientů jsem rekurzivně spustil
stejnou funkci (to však nakonec nebylo zapotřebí). Zde jsou níže
nalezené funkce:

\begin{verbatim}
  e(a, b, c, d) = 5a + 0b + 3c + 4d % 7
  f(a, b) = 5a + 1b % 7
  g(a, b, c) = 5a + 0b + 4c % 7
  h(a, b, c) = 2a + 2b + 4c % 7
  i(a, b, c) = 0a + 3b + 2c % 7
  j(a, b, c, d) = 3a + 3b + 6c + 1d % 7
  k(a, b, c, d) = 5a + 3b + 4c + 3d % 7
  l(a, b) = 5a + 4b % 7
  m(a, b, c, d) = 5a + 6b + 1c + 0d % 7
  n(a, b) = 4a + 1b % 7
\end{verbatim}

Tím už můžeme vyřešit velkou instanci:

\begin{lstlisting}
  g(g(1,6,0),f(6,n(3,5)),i(l(3,4),4,0)) = 3
  m(5,m(f(5,3),6,f(k(3,2,4,0),6),3),l(l(4,2),5),h(f(1,5),h(2,2,5),6)) = 0
  m(n(e(6,f(1,6),0,6),n(g(5,2,1),3)),i(3,m(l(5,2),1,0,2),i(4,k(4,0,1,5),e(5,3,5,6))),n(2,n(l(5,0),3)),m(m(5,5,0,5),j(2,3,3,1),f(5,2),0)) = 6
  i(i(h(6,4,4),l(n(6,2),4),m(h(0,0,4),l(1,3),3,4)),1,e(j(l(6,0),1,0,i(1,2,1)),k(2,4,2,5),e(m(2,3,4,2),k(2,5,1,5),g(0,3,1),4),6)) = 5
  l(i(k(n(0,0),6,5,2),4,f(6,5)),j(1,j(6,0,4,k(5,0,5,3)),5,1)) = 4
  e(0,m(g(0,5,f(4,1)),j(2,k(5,0,4,3),e(2,6,6,5),0),6,g(4,4,4)),5,g(n(e(4,3,0,0),0),1,4)) = 6
  k(i(3,1,k(h(1,1,5),j(5,6,4,4),5,5)),2,m(1,5,2,n(3,i(5,4,3))),m(3,1,5,m(e(5,1,3,0),4,h(4,3,4),6))) = 1
  m(f(0,e(2,l(4,6),0,2)),g(3,e(0,3,e(3,4,1,6),3),n(j(4,3,1,0),5)),2,j(l(5,3),6,i(5,3,0),6)) = 3
  k(h(2,1,1),i(1,4,i(5,n(6,1),1)),j(g(0,3,1),3,6,4),l(n(f(0,3),1),k(m(6,6,5,4),5,0,0))) = 2
  i(g(4,2,3),i(e(0,2,3,5),4,h(0,1,4)),f(l(5,6),n(j(5,5,6,6),4))) = 0
  h(n(h(1,2,6),2),4,k(3,4,4,e(5,j(4,4,5,6),h(3,6,4),4))) = 5
  k(m(3,3,h(2,1,3),h(3,3,2)),m(4,g(2,1,2),f(4,5),n(4,5)),4,m(3,0,l(0,k(4,4,1,1)),l(l(1,0),4))) = 4
  f(1,m(3,j(h(3,2,2),1,0,1),0,3)) = 4
  f(n(m(k(0,1,5,3),0,0,1),h(g(6,4,0),1,5)),m(n(4,5),j(1,6,3,e(3,5,5,0)),m(1,5,0,2),4)) = 6
  j(n(h(h(5,2,4),4,2),k(5,0,4,6)),i(5,m(2,i(6,1,2),n(4,4),i(5,0,1)),4),g(g(2,1,h(6,1,5)),g(i(2,3,1),2,n(5,6)),3),g(l(k(0,0,3,6),0),3,l(e(2,0,4,3),g(1,0,0)))) = 2
  f(i(5,4,m(j(4,6,4,6),5,1,5)),l(0,5)) = 2
  h(l(6,2),3,l(5,3)) = 6
  m(l(0,5),l(5,i(3,l(2,0),5)),l(e(6,2,2,m(6,6,0,2)),4),f(0,4)) = 3
  e(n(i(f(2,2),4,0),m(m(1,2,5,5),3,n(3,6),m(5,0,3,1))),n(f(6,3),1),4,h(4,2,e(4,1,5,m(2,0,2,1)))) = 6
  f(5,4) = 1
  k(n(2,k(4,f(5,5),6,4)),5,k(k(0,1,0,3),0,l(g(4,5,4),3),0),0) = 2
  m(3,f(m(0,4,0,k(4,6,2,0)),3),4,k(i(5,4,g(6,3,6)),h(5,4,2),j(0,3,4,4),e(2,5,0,h(6,2,5)))) = 1
  g(j(6,3,1,1),3,l(0,n(f(5,0),1))) = 1
  m(3,0,j(e(6,6,4,l(3,0)),3,2,0),e(e(6,0,0,6),i(g(2,0,4),3,3),4,i(6,1,2))) = 6
  l(l(n(4,2),i(5,i(2,0,2),h(2,4,0))),e(e(4,1,e(1,1,6,3),m(1,0,5,5)),f(0,n(0,6)),h(1,3,3),6)) = 4
  l(0,f(l(g(1,1,2),j(3,4,2,4)),5)) = 3
  l(f(m(3,e(5,3,0,2),2,4),2),k(n(3,2),6,j(2,5,4,n(0,6)),1)) = 6
  i(h(e(0,k(2,5,3,2),i(5,1,4),0),2,1),k(h(3,1,0),n(4,5),5,g(g(3,1,2),3,3)),i(l(3,f(0,6)),n(2,0),1)) = 3
  h(l(4,l(g(6,5,4),2)),k(i(5,0,4),l(6,6),5,n(1,2)),k(3,e(2,m(5,1,3,1),2,0),i(i(3,0,0),5,h(0,2,2)),l(2,j(0,2,4,2)))) = 0
  l(i(n(6,1),j(4,6,6,3),k(3,1,1,n(3,3))),0) = 4
  l(k(l(1,3),4,0,h(6,3,1)),2) = 4
  h(f(6,g(4,5,2)),h(1,2,n(1,3)),h(e(3,4,3,l(1,2)),n(4,0),f(2,6))) = 0
  n(i(j(2,6,4,6),n(3,0),n(6,0)),3) = 3
  g(2,f(g(h(4,5,1),5,0),5),l(h(1,3,0),3)) = 1
  j(j(e(2,2,6,k(6,0,4,4)),g(6,4,n(5,2)),6,k(h(1,1,1),4,n(2,6),5)),l(0,3),e(1,4,f(0,n(4,6)),2),g(i(4,1,k(3,3,0,5)),6,j(1,6,2,0))) = 5
  m(1,0,4,m(f(3,3),0,4,4)) = 2
  l(h(1,m(3,1,j(4,5,1,2),0),n(l(1,4),1)),2) = 3
  h(g(l(m(3,1,0,3),2),j(2,5,2,e(5,2,3,1)),l(e(6,6,6,4),2)),e(0,4,i(l(0,3),0,4),k(1,1,1,1)),j(l(3,0),6,g(m(2,4,3,1),1,2),e(1,4,h(0,4,6),5))) = 1
  l(i(1,0,0),j(j(1,5,6,5),f(5,3),k(5,6,0,e(1,6,2,6)),g(k(6,0,5,0),4,1))) = 5
  k(4,i(4,4,1),e(5,k(4,6,m(5,2,0,4),5),e(4,3,4,1),m(2,3,6,0)),5) = 5
  f(m(g(4,1,5),1,k(0,k(1,4,4,3),4,f(1,3)),1),k(f(1,2),l(4,5),3,e(h(6,3,4),5,f(5,2),3))) = 2
  k(k(6,0,l(4,i(1,5,6)),h(4,0,3)),1,6,i(g(3,1,3),l(g(3,5,0),1),f(m(4,5,6,6),f(6,1)))) = 0
  g(n(k(6,j(4,2,0,3),0,6),m(1,6,5,3)),k(6,m(5,4,5,2),5,i(5,m(1,0,3,4),5)),f(l(1,2),j(3,1,5,3))) = 6
  f(i(5,0,3),0) = 2
  l(k(2,n(3,i(6,3,0)),1,m(n(0,1),1,h(5,1,0),j(6,4,3,3))),3) = 0
  g(n(k(6,3,3,f(3,1)),3),k(g(k(6,3,5,2),i(2,2,1),1),i(0,5,0),i(h(5,6,3),4,4),6),m(e(k(3,3,2,5),3,3,5),j(2,5,5,3),g(5,g(1,5,4),m(1,3,2,3)),3)) = 6
  n(m(n(1,m(3,1,2,3)),j(2,4,m(6,3,6,3),n(3,5)),3,k(0,i(2,2,3),1,k(1,4,0,0))),e(i(3,6,6),i(6,6,l(5,5)),l(m(4,4,4,5),m(2,5,3,2)),0)) = 5
  m(6,i(0,5,5),l(l(4,0),3),m(5,f(h(4,5,2),3),f(3,2),3)) = 5
  f(4,e(i(0,6,j(6,1,0,0)),0,1,g(5,2,1))) = 5
  i(f(2,5),k(l(m(6,4,2,2),6),k(1,3,n(6,2),l(1,1)),4,k(3,6,0,6)),l(5,6)) = 2
\end{lstlisting}

\end{document}
