\documentclass{fkssolpub}

\usepackage[czech]{babel}
\usepackage{fontspec}
\usepackage{listings}
\usepackage{color}
\definecolor{codegray}{rgb}{0.5,0.5,0.5}
\lstset{
  basicstyle=\ttfamily,
  breaklines=true,
  keywordstyle=\color{blue},
  numberstyle=\color{codegray}\ttfamily,
  numbers=left,
  }

\author{Ondřej Sedláček}
\school{Gymnázium Oty Pavla} 
\series{4}
\problem{3} 

\begin{document}

Pokud bychom směli používat knihovní funkce, pak lze tento úkol
celkem jednoduše vyřešit pomocí jedné funkce v \verb|scipy|, která
umožňuje jednoduše použít jedné optimalizační techniky se jménem
lineární programování. Díky ní jsme schopni vyřešit soustavu linárních
rovnic (což tento úkol jde takto interpretovat) a pak najít řešení,
jejíž hodnota cenové funkce bude co nejnižší (či nejvyšší). Tudíž
naším jediným úkolem by bylo zadat matici $A$, jejíž sloupce budou
tvořit množství tuku a bílkovin jednotlivých mas, pak cenový vektor
$\mathbf{c}$ jako seznam cen jednotlivých mas a požadované množství
tuku a bílkovin $\mathbf{b}$.

\begin{lstlisting}[language=Python]
import numpy as np
from scipy.optimize import linprog

def krmeni(A, b, c):
    bounds = [(0, None) for _ in range(A.shape[1])]
    result = linprog(c=c, bounds=bounds, A_eq=A, b_eq=b)
    return result.x
\end{lstlisting}

Pokud však budeme chtít řešit bez takové knihovny, je velmi užitečné
si uvědomit, že stačí pro dosažení minima koupit jenom ze dvou druhů
mas, které v daný moment budou nejvýhodnější. Je to dáno tím, že
když kupujeme více druhů mas, tak jeden druh masa lze nahradit, aniž
by cena vzrostla.

Tím můžeme velmi přímočaře vyřešit tento problém následujícím 
způsobem -- na začátku seřadíme druhy masa podle poměru tuku vůči
bílkovinám, pak pro každou šelmu rozdělíme druhy masa podle toho,
jestli má pro šelmu příliš tuků či bílkovin, a pro každou dvojici
mezi těmito dvěmi skupinami vyřešíme systém lineárních rovnic
a zjistíme nejlepší řešení. Složitost tohoto algoritmu pak bude
$\mathcal{O}(K N^2)$.

\end{document}
