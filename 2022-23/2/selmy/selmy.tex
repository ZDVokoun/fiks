\documentclass{fkssolpub}

\usepackage[czech]{babel}
\usepackage{fontspec}

\author{Ondřej Sedláček}
\school{Gymnázium Oty Pavla} 
\series{2}
\problem{1} 

\begin{document}

\section{První úkol}

Nejdříve popíši situaci pro všechny topologie bez šelm. Když v každé z těchto
topologií nebude ani jedna šelma, pak, jak z definice jednotlivých topologií 
vyplývá, můžeme dojít na každý sloup skalního města. Proto pro každou z těchto topologií
platí, že když se ve skalním městě nebude nacházet ani jedna šelma, pak
jsme schopni úkol splnit vždycky.

Dále rozeberu jednotlivé topologie s přítomností šelem. Budu předpokládat, že
hráč, šelmy i stěžeň budou na různých sloupech.

Z definice první topologie můžeme zjistit, že v první topologii je graf
skalního města strom. Abychom dokázali, že nejsme schopni úkol splnit
v každém případě, musíme dokázat, že po odstranění jednoho či dvou vrcholů
grafu skalního města vzniknout nové komponenty. Protože v této topologii
je grafem strom, stane se tak vždy, když je stupeň odstraňovaného vrcholu
alespoň 2 (nejedná se o list). Tím jsme dokázali, že v tomto případě není
možné úkol splnit nezávisle na rozmístění.

Pokud u druhé topologie počítáme jen s jednou šelmou, pak úkol jsme schopni
splnit vždycky, protože graf skalního města této topologie neobsahuje žádné
artikulace. Avšak pokud jsou dvě šelmy, mohou být rozmístěny tak, že izolují
sloup v rohu, protože tento sloup má jenom dva sousedy. Proto není možné
pro tuto topologii a dvě šelmy splnit úkol nezávisle na rozmístění, avšak pro
jednu šelmu to lze.

U třetí topologie musíme nejdříve rozebrat malá skalní města. Skalní města
$Q_0$ a $Q_1$ jsou tak malá, že není možné splnit předpoklad v druhém
odstavci. Kdybychom se je tam přesto pokoušeli rozmístit, nemohli bychom v
ani jednom případě splnit úkol. Skalní město $Q_2$ lze zařadit do druhé
topologie, proto v tomto případě také nelze splnit úkol nezávisle na
rozmístění. 

Avšak pro skalní město $Q_3$ už je to jiné. Pro izolaci komponenty
o jednom vrcholu musíme odstranit nejméně tři vrcholy a o dvou nebo třech vrcholech
musíme odstranit nejméně čtyři. Z čehož vyplývá, že jedna nebo dvě šelmy
nejsou schopny skalní město rozdělit na komponenty, proto jsme pro toto
skalní město splnit úkol nezávisle na rozmístění. Tento závěr můžeme přenést
i pro skalní města $Q_n$ pro $n > 2$, protože díky definici třetí topologie
nemůže nastat, aby pro skalní město $Q_k$ bylo možné splnit úkol vždy a pro
skalní město $Q_{k+1}$ tomu bylo naopak.

\section{Druhý úkol}

Začnu nejprve druhou topologií a jednou šelmou. Protože v tomto případě
se nemůže hráč dostat do takové situace, kdy hráč se nemůže dostat do bezpečné
vzdálenosti od šelmy, strategie hráče na utíkání je jednoduchá -- jakmile
hráč je na sloupu vedle sloupu se šelmou, posune se na sloup, která nesousedí
se sloupem s šelmou.

Se dvěmi šelmami je to přesně naopak, šelmy hráče vždy v nějakém momentu
dostanou. Šelmám stačí jen se dostat do polohy, kdy jedna šelma sousedí s hráčem
a druhá sousedí s tou první šelmou a zároveň je "šikmo" na sloup s hráčem.
Tehdy můžou hráče zatlačit do rohu a obklíčit, protože tehdy omezí pohyb hráče
jen do dvou směrů a zároveň po pohybu hráče jsou šelmy schopny se do tohoto
rozpoložení vrátit. Zároveň není žádná strategie pro hráče, aby znemožnil
použití této strategie šelmami.

Pro třetí topologii s jednou šelmou se to liší podle velikosti. Pro skalní
města $Q_0$ a $Q_1$ je zřejmé, že šelma vždy vyhraje. Skalní město
$Q_2$ odpovídá i druhé topologii, proto zde hráči stačí stejná strategie
jako pro druhou topologii s jednou šelmou, a ve větších skalních městech
třetí topologie taky stačí tatáž strategie, protože díky definici pro tato
města nemůže nastat situace, kdy hráč bude obklíčen jednou šelmou.

Se dvěmi šelmami se k skalním městům, kdy hráč nemůže utéct, přidá $Q_2$,
protože $Q_2$ patří i do druhé topologie, a $Q_3$, protože šelmám stačí se
rozmístit do takových vrcholů, jejichž hrana, kdyby existovala, by při 
znázornění skalního města krychlí tvořila úhlopříčku krychle, aby 
hráče obklíčili. Pro $Q_4$ však stačí hráči se pohybovat v bezpečné
vzdálenosti od šelem, protože pro toto skalní město nemůžou šelmy hráče
obklíčit. Tohle díky definici třetí topologie bude platit i pro větší
skalní města této topologie.

Pro čtvrtou topologii s jednou šelmou záleží na skalním městě a poloze
šelmy, jestli hráč bude schopen utíkat do nekonečna. Pokud se ve městě
nachází jenom cykly o třech vrcholech anebo je šelma schopna zablokovat
přístup k větším cyklům předtím, než se tam hráč dostane, pak šelma
může obklíčit hráče prostým přibližováním se. Pokud ale hráč má možnost
se dostat na cyklus o alespoň čtyřech vrcholech, pak hráči stačí se vždy
posouvat na vrchol v cyklu v bezpečné vzdálenosti od šelmy. Proto aby
mohl hráč utíkat do nekonečna, musí platit, že se ve městě nachází
alespoň jeden cyklus o alespoň čtyřech vrcholech a vzdálenost od hráče k
nejbližšího takového cyklu je menší nebo rovno vzdálenosti od šelmy
k takovému cyklu. V ostatních případech šelma vždy hráče obklíčí a vyhraje.

Když jsou ale dvě šelmy, pak můžou znemožnit nekonečný útěk hráče tím,
že po tom, co se dostanou do cyklu, se rozdělí. Proto výherní strategie
pro šelmy je taková, že pokud se nenachází na cyklu nebo se nachází na
spojnici cyklů, pak jdou společně, a pokud jdou procházet cyklem, pak
se rozdělí a jakmile se dostanou rozcestí, kudy míří cesta k hráči, počkají
na sebe. Takhle pak následují hráče do té doby, dokud ho neobklíčí.

\end{document}
